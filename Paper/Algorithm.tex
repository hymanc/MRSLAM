\section{The Algorithm}
\label{S:Alg}

\subsection{Overview}
\label{SS:Alg:Overview} 


Howard’s multi-robot SLAM algorithm generates a single map and pose posterior much in the same way an occupancy grid map SLAM algorithm based on a Rao-Blackwellized particle filter (RBPF) would, but with additions to accommodate multiple robots. Mapping begins with a set consisting of a single robot with known pose and an occupancy grid map of the environment is built as this robot traverses and measures it. The ultimate goal of the algorithm is to simultaneously compute for time $t$, the full SLAM posterior containing all robot pose trajectories $x_{1:t}^i$ and the global map $m_{1:t}$, given only a single known initial pose and sets of measurements
$$p(x_{1:t}^1,x_{1:t}^2,...,x_{1:t}^M,m_{1:t}|x_0^1,z_{1:t}^1,u_{0:t}^1,\Delta_s^2,z_{1:t}^2,u_{0:t}^2,\Delta_s^3,...,\Delta_s^M,z_{1:t}^M,u_{0:t}^M)$$

The odometry and measurement data of all other robots is stored as it is collected, as it cannot contribute to the global map without a known relative pose linking it to the frame of the first robot. As the first robot encounters additional robots via a mutual pose observation, the newly observed robot is added to the set of mapping robots and its actions and the map is sequentially conditioned by its actions and measurements. From the point of observation, all future actions and measurements of the new robot are used to condition the map posterior, and likewise all of its previously stored actions and measurements are played back in reverse order from that point as a virtual robot. This information is then passed to a pair of new particle filters for the pose of these robots that contribute to the same global map. 
As further encounters occur between any robot, real or virtual, in the mapping set and a previously unseen robot, the previously unseen robot is added to the set of mapping robots and its measured relative pose is again used to establish a reference point with which its actions and measurements can condition the map posterior. Further mutual observations of two robots already in the mapping set are ignored for the sake of simplicity. This encounter-add process is continued recursively for all robots until all stored and future odometry and measurements are exhausted or all possible mapping robots have made an encounter with the mapping set. Beyond this point, the algorithm behaves as a normal RBPF with a stacked state containing pose posteriors of all robots in addition to the map.


    This algorithm relies on a number of assumptions that are requisite for it to effectively solve the MRSLAM problem. Firstly, for all explored regions to count towards the map, each mapping robot must have encountered a robot that is an element of the mapping set in order to make a fully connected graph of robot poses, and therefore maximally complete map. Additionally, each robot pose trajectory is independent of all other trajectories such that motion or observation from one robot does not affect another outside of encounter events \cite{howard2006multi}.
